\documentclass[a4paper,11pt]{jsarticle}
\usepackage{amsmath,amssymb} %\dfrac
\usepackage{bm} %\bm:斜体太字の文字を書けるようにする(ベクトル表記)
\usepackage{ascmac} %参考http://akita-nct.jp/yamamoto/comp/latex/make_doc/box/box.php
\usepackage{fancyhdr} %ヘッダに任意の文字列を書く
\usepackage{amsfonts} %\mathbb:白抜き文字を書けるようにする
\usepackage{enumerate}

\usepackage[dvipdfmx]{graphicx} %図表を入れる場合
\usepackage{float} %図表の位置
\usepackage{subfigure}

\makeatletter
   \renewcommand{\theequation}{%
   \thesection.\arabic{equation}}
   \@addtoreset{equation}{section}
\makeatother

\newtheorem{Definition}{定義}[section]
\newtheorem{Theorem}{定理}[section]
\newtheorem{Lemma}{補助定理}[section]
\newtheorem{Proof}{証明}[section]

%偏微分,チルダ,分数
\newcommand{\pd}[2]{\dfrac{\partial #1}{\partial #2}}
\newcommand{\pdd}[2]{\dfrac{{\partial}^2 #1} {\partial {#2}^2} }
\newcommand{\ti}[1]{\tilde{#1}}
\newcommand{\df}[2]{\dfrac{#1}{#2}}
\newcommand{\intinf}{\int_{-\infty}^{\infty}}
\newcommand{\intzero}{\int_{0}^{\infty}}
\newcommand{\E}{\mathrm{E}}
\newcommand{\F}{\mathcal{F}}

\title{Solutions Manual for Shreve2}
\author{Kohei Fukushima}
\date{\today}

\begin{document}
\maketitle
\pagestyle{fancy}
\lhead{Kohei Fukushima}
\rhead{\today}
%\quad \; \: \, 左ほど大きな空白
%\cdot ・ドット
\section{ブラウン運動}
\noindent
3.1\\
Proof.
$0 \leq t < u_1 < u_2$に対して,
$\mathcal{F}(t)\subset\mathcal{F}(u_1)$(定義3.3.3(i)情報の蓄積)がいえる.
また,$W(u_2)-W(u_1)$は$\mathcal{F}(u_1)$
と独立(定義3.3.3(iii)将来の増分の独立性)である.
以上より,$W(u_2)-W(u_1)$は$\F_t$と独立である.
\\ \\
3.2
\begin{align*}
  \begin{split}
    W_t^2-t&=\{(W_t-W_s)+W_s\}^2-t \\
    &=(W_t-W_s)^2+2(W_t-W_s)W_s+W_s^2-t,
  \end{split}
\end{align*}
\begin{align*}
  \begin{split}
    \E[W_t^2-t|\mathcal{F}_s]
    &=\E[(W_t-W_s)^2|\mathcal{F}_s]
    +2\E[(W_t-W_s)W_s|\mathcal{F}_s]
    +\E[W_s|\mathcal{F}_s]-t \\
    &=\E[(W_t-W_s)^2]+2\E[W_t-W_s]\E[W_s|\F_s]+\E[W_s^2|\F_s]-t\\
    &=(t-s)+2\cdot0\cdot W_s+W_s^2-t=W_s^2-s.
  \end{split}
\end{align*}
\\ \\
3.3\\
$X-\mu \, \sim N(0,\sigma^2)$のモーメント母関数とその微分は,
\begin{align*}
  &\varphi(u)=\E[e^{u(X-\mu)}]=e^{\sigma^2 u^2/2}, \,
  \varphi^{(1)}(u)=\sigma^2 u e^{\sigma^2 u^2/2}, \,
  \varphi^{(2)}(u)=(\sigma^2 + \sigma^4 u^2) e^{\sigma^2 u^2/2}, \\
  &\varphi^{(3)}(u)=(2\sigma^4 u + \sigma^4 u +\sigma^6 u^3)e^{\sigma^2 u^2/2}
  =(3\sigma^4 u + \sigma^6 u^3)e^{\sigma^2 u^2/2}, \\
  &\varphi^{(4)}(u)=(3\sigma^4 + 3\sigma^6 u^2 +3\sigma^6 u^2+ \sigma^8 u^4)
  e^{\sigma^2 u^2/2}
  =(3\sigma^4 + 6\sigma^6 u^2+ \sigma^8 u^4)e^{\sigma^2 u^2/2}.
\end{align*}
尖度=$\E[(X-\mu)^4]=\varphi^{(4)}(0)=3\sigma^4$.
\\ \\
3.4 ブラウン運動の他の変分(variation) \\
(i)
以下を満たすような集合$A\in\F$が存在すると仮定する:\\
$\mathrm{P}(A)>0, \, \omega \in A, \,
\lim_{|\Pi| \to 0}\sum_{j=0}^{n-1}|W_{t_{j+1}}-W_{t_{j}}|(\omega) < \infty$ \\
$\sum_{j=0}^{n-1}(W_{t_{j+1}}-W_{t_{j}})^2(\omega) \leq
\max_{0\leq k\leq n-1}|W_{t_{k+1}}-W_{t_{k}}|(\omega)\cdot
\sum_{j=0}^{n-1}|W_{t_{j+1}}-W_{t_{j}}|(\omega)$ となるが,\\
$|\Pi| \to 0$とするとき,
左辺はT,右辺は,$0\cdot(有限)=0$となり,矛盾する.
したがって,1次変分はほとんどすべての経路で無限大に発散する.\\
(ii)
$\sum_{j=0}^{n-1}(W_{t_{j+1}}-W_{t_{j}})^3 \leq
\max_{0\leq k\leq n-1}|W_{t_{k+1}}-W_{t_{k}}|\cdot
\sum_{j=0}^{n-1}(W_{t_{j+1}}-W_{t_{j}})^2$は,\\
$|\Pi|\to 0$のとき,
$\to 0\cdot T=0$
\\ \\
3.5 ブラック-ショールズ-マートンの公式
\begin{align*}
  \begin{split}
    \E[e^{-rT}(S_T-K)^+]
    &=e^{-rT}\int_{w\in D} (S_0e^{(r-\sigma^2/2)T+\sigma w}-K)
    \df{1}{\sqrt{2\pi T}}e^{-w^2/2T} dw \\
    &=e^{-rT}\int_{\sqrt{T}z\in D} (S_0e^{(r-\sigma^2/2)T+\sigma \sqrt{T}z}-K)
    \df{1}{\sqrt{2\pi}}e^{-z^2/2} dz \\
    &=S_0 \int_{\sqrt{T}z\in D}
    \df{1}{\sqrt{2\pi}}e^{-(z-\sigma \sqrt{T})^2/2} dz
    -K e^{-rT} \int_{\sqrt{T}z\in D}
    \df{1}{\sqrt{2\pi}}e^{-z^2/2} dz \\
    &=S_0 \int_{\sqrt{T}(y+\sigma\sqrt{T})\in D}\phi(y)dy
    -K e^{-rT} \int_{\sqrt{T}z\in D}\phi(z)dz
  \end{split}
\end{align*}
ここで,
\begin{align*}
  D=\{x;S_T(x)-K>0\}=\left\{x; x>\df{1}{\sigma} \left[\ln{\df{K}{S_0}}
  -\left(r-\df{\sigma^2}{2}\right) T \right] \right\}
\end{align*}
であるから,最終行の$y,z$の積分範囲はそれぞれ,
\begin{align*}
  &y>\df{1}{\sigma\sqrt{T}} \left[\ln{\df{K}{S_0}}
  -\left(r-\df{\sigma^2}{2}\right) T \right] - \sigma\sqrt{T}
  %=\df{1}{\sigma\sqrt{T}} \left[\ln{\df{K}{S_0}}
  %-\left(r+\df{\sigma^2}{2}\right) T \right]
  =-\df{1}{\sigma\sqrt{T}} \left[\ln{\df{S_0}{K}}
  +\left(r+\df{\sigma^2}{2}\right) T \right] \\
  & z>\df{1}{\sigma\sqrt{T}} \left[\ln{\df{K}{S_0}}
  -\left(r-\df{\sigma^2}{2}\right) T \right]
  =-\df{1}{\sigma\sqrt{T}} \left[\ln{\df{S_0}{K}}
  +\left(r-\df{\sigma^2}{2}\right) T \right]
\end{align*}
であり,
\begin{align*}
  d_{\pm}(T,S_0)=\df{1}{\sigma\sqrt{T}} \left[\ln{\df{S_0}{K}}
  +\left(r\pm\df{\sigma^2}{2}\right) T \right]
\end{align*}
とおけば,
\begin{align*}
  \E[e^{-rT}(S_T-K)^+]=S_0 N(d_+(T,S_0))
  -K e^{-rT} N(d_-(T,S_0))
\end{align*}
である.
\\ \\
3.6 \\
(i)
ドリフトを持つブラウン運動$X_t=\mu t+W_t$について,
\begin{align*}
  \begin{split}
    \E[f(X_t)|\F(s)]&=\E[f(\mu t+W_t)|\F(s)]
    =\E[f(W_t-W_s+W_s+\mu t)|\F(s)] \\
    &=\intinf f(w+W_s+\mu t)\df{1}{\sqrt{2\pi(t-s)}}
    \exp\left(-\frac{w^2}{2(t-s)}\right)dw \\
    &=\intinf f(y)\df{1}{\sqrt{2\pi(t-s)}}
    \exp\left(-\frac{(y-W_s-\mu t)^2}{2(t-s)}\right)dy\\
    &=\intinf f(y)\df{1}{\sqrt{2\pi(t-s)}}
    \exp\left(-\frac{(y-W_s-\mu s-\mu(t-s))^2}{2(t-s)}\right)dy\\
    &=\intinf f(y)\df{1}{\sqrt{2\pi(t-s)}}
    \exp\left(-\frac{(y-X_s-\mu(t-s))^2}{2(t-s)}\right)dy\\
    &=\intinf f(y)p(t-s,X_s,y)dy=g(X_s).
  \end{split}
\end{align*}
\\
(ii)
$\mu=\nu/\sigma$とすれば,
幾何ブラウン運動は,
$S_t=S_0 e^{\sigma W_t+\nu t}=S_t=S_0 e^{\sigma X_t}$とかける.
(i)の結果(5行目)を利用して,
\begin{align*}
  \begin{split}
    \E[f(S_t)|\F(s)]&=\E[f(S_0 e^{\sigma X_t})|\F(s)] \\
    &=\intinf f(S_0 e^{\sigma y})\df{1}{\sqrt{2\pi(t-s)}}
    \exp\left(-\frac{(y-X_s-\mu(t-s))^2}{2(t-s)}\right)dy
  \end{split}
\end{align*}
$z=S_0 e^{\sigma y}=S_0 e^{\sigma X_t}$という変数変換を考えれば,
積分範囲は$z:0\to\infty$で,\\
$y=\df{1}{\sigma}\ln{\df{z}{S_0}}, \quad dy=\df{1}{\sigma z}dz, \quad
y-X_s=X_t-X_s=\df{1}{\sigma}(\ln{\df{z}{S_0}}-\ln{\df{S_s}{S_0}})
=\df{1}{\sigma}\ln{\df{z}{S_s}}$であるから,
\begin{align*}
  \begin{split}
    \E[f(S_t)|\F(s)]
    &=\intzero f(z)\df{1}{\sqrt{2\pi(t-s)}}
    \exp\left\{-\frac{\left( \df{1}{\sigma}\ln\df{z}{S_s}
    -\mu(t-s) \right)^2}{2(t-s)}\right\}\df{1}{\sigma z}dz \\
    &=\intzero f(z)\df{1}{\sigma z\sqrt{2\pi(t-s)}}
    \exp\left\{-\frac{\left(\ln(z/S_s)-\nu(t-s)\right)^2}
    {2\sigma^2(t-s)}\right\}dz \\
    &=\intzero f(z) p(t-s,S_s,z)dz=g(S_s).
  \end{split}
\end{align*}
\end{document}
