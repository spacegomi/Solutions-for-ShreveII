\documentclass[a4paper,11pt]{jsarticle}
\usepackage{amsmath,amssymb} %\dfrac
\usepackage{bm} %\bm:斜体太字の文字を書けるようにする(ベクトル表記)
\usepackage{ascmac} %参考http://akita-nct.jp/yamamoto/comp/latex/make_doc/box/box.php
\usepackage{fancyhdr} %ヘッダに任意の文字列を書く
\usepackage{amsfonts} %\mathbb:白抜き文字を書けるようにする
\usepackage{enumerate}

\usepackage[dvipdfmx]{graphicx} %図表を入れる場合
\usepackage{float} %図表の位置
\usepackage{subfigure}

\makeatletter
   \renewcommand{\theequation}{%
   \thesection.\arabic{equation}}
   \@addtoreset{equation}{section}
\makeatother

\newtheorem{Definition}{定義}[section]
\newtheorem{Theorem}{定理}[section]
\newtheorem{Lemma}{補助定理}[section]
\newtheorem{Proof}{証明}[section]

%偏微分,チルダ,分数
\newcommand{\pd}[2]{\dfrac{\partial #1}{\partial #2}}
\newcommand{\pdd}[2]{\dfrac{{\partial}^2 #1} {\partial {#2}^2} }
\newcommand{\ti}[1]{\tilde{#1}}
\newcommand{\df}[2]{\dfrac{#1}{#2}}
\newcommand{\intinf}{\int_{-\infty}^{\infty}}
\newcommand{\intzero}{\int_{0}^{\infty}}
\newcommand{\E}{\mathrm{E}}
\newcommand{\F}{\mathcal{F}}
\newcommand{\1}{\mbox{1}\hspace{-0.30em}\mbox{1}}

\title{Solutions Manual for ShreveII}
\author{spacegomi}
\date{}

\begin{document}
\maketitle
\pagestyle{fancy}
\lhead{Solutions Manual for ShreveII}
%\rhead{\today}
%\quad \; \: \, 左ほど大きな空白
%\cdot ・ドット
\setcounter{section}{2}
\section{ブラウン運動}
\subsection{}
%\noindent
$0 \leq t < u_1 < u_2$に対して,
$\mathcal{F}(t)\subset\mathcal{F}(u_1)$(定義3.3.3(i)情報の蓄積)がいえる.
また,$W(u_2)-W(u_1)$は$\mathcal{F}(u_1)$
と独立(定義3.3.3(iii)将来の増分の独立性)である.
以上より,$W(u_2)-W(u_1)$は$\F_t$と独立である.

\subsection{} %3.2
$0\leq s\leq t$に対して,
\begin{align*}
  \begin{split}
    W_t^2-t&=\{(W_t-W_s)+W_s\}^2-t \\
    &=(W_t-W_s)^2+2(W_t-W_s)W_s+W_s^2-t,
  \end{split}
\end{align*}
とおく.
\begin{align*}
  \begin{split}
    \E[W_t^2-t|\mathcal{F}_s]
    &=\E[(W_t-W_s)^2|\mathcal{F}_s]
    +2\E[(W_t-W_s)W_s|\mathcal{F}_s]
    +\E[W_s|\mathcal{F}_s]-t \\
    &=\E[(W_t-W_s)^2]+2\E[W_t-W_s]\E[W_s|\F_s]+\E[W_s^2|\F_s]-t\\
    &=(t-s)+2\cdot0\cdot W_s+W_s^2-t=W_s^2-s.
  \end{split}
\end{align*}
以上より$W_t^2-t$はマルチンゲールである.

\subsection{} %3.3
$X-\mu \, \sim N(0,\sigma^2)$のモーメント母関数とその微分は,
\begin{align*}
  &\varphi(u)=\E[e^{u(X-\mu)}]=e^{\sigma^2 u^2/2}, \,
  \varphi^{(1)}(u)=\sigma^2 u e^{\sigma^2 u^2/2}, \,
  \varphi^{(2)}(u)=(\sigma^2 + \sigma^4 u^2) e^{\sigma^2 u^2/2}, \\
  &\varphi^{(3)}(u)=(2\sigma^4 u + \sigma^4 u +\sigma^6 u^3)e^{\sigma^2 u^2/2}
  =(3\sigma^4 u + \sigma^6 u^3)e^{\sigma^2 u^2/2}, \\
  &\varphi^{(4)}(u)=(3\sigma^4 + 3\sigma^6 u^2 +3\sigma^6 u^2+ \sigma^8 u^4)
  e^{\sigma^2 u^2/2}
  =(3\sigma^4 + 6\sigma^6 u^2+ \sigma^8 u^4)e^{\sigma^2 u^2/2}.
\end{align*}
尖度=$\E[(X-\mu)^4]=\varphi^{(4)}(0)=3\sigma^4$.

\subsection{ブラウン運動の他の変分(variation)} %3.4
\subsubsection{(i)}
以下を満たすような集合$A\in\F$が存在すると仮定する:\\
\begin{align*}
  \Pr(A)>0, \, \omega \in A, \,
  \lim_{|\Pi| \to 0}\sum_{j=0}^{n-1}|W_{t_{j+1}}-W_{t_{j}}|
  (\omega) < \infty
\end{align*}
このとき,
\begin{align*}
  \sum_{j=0}^{n-1}(W_{t_{j+1}}-W_{t_{j}})^2(\omega) \leq
  \max_{0\leq k\leq n-1}|W_{t_{k+1}}-W_{t_{k}}|(\omega)\cdot
  \sum_{j=0}^{n-1}|W_{t_{j+1}}-W_{t_{j}}|(\omega)
\end{align*}
となるが,\\
$|\Pi| \to 0$とするとき,
左辺はT,右辺は,$0\cdot(有限)=0$となり,矛盾する.
したがって,1次変分はほとんどすべての経路で無限大に発散する.

\subsubsection{(ii)}
3次変分について,以下の不等式を得る.
\begin{align*}
  \sum_{j=0}^{n-1}(W_{t_{j+1}}-W_{t_{j}})^3 \leq
  \max_{0\leq k\leq n-1}|W_{t_{k+1}}-W_{t_{k}}|\cdot
  \sum_{j=0}^{n-1}(W_{t_{j+1}}-W_{t_{j}})^2
\end{align*} 
この式の右辺は,$|\Pi|\to 0$とするとき,
$0\cdot T=0$となる.

\subsection{ブラック-ショールズ-マートンの公式} %3.5
$\phi(x)$は標準正規分布の確率密度関数を表すものとする.
\begin{align*}
  \begin{split}
    \E[e^{-rT}(S_T-K)^+]
    &=e^{-rT}\int_{w\in D} (S_0e^{(r-\sigma^2/2)T+\sigma w}-K)
    \df{1}{\sqrt{2\pi T}}e^{-w^2/2T} dw \\
    &=e^{-rT}\int_{\sqrt{T}z\in D} (S_0e^{(r-\sigma^2/2)T+\sigma \sqrt{T}z}-K)
    \df{1}{\sqrt{2\pi}}e^{-z^2/2} dz \\
    &=S_0 \int_{\sqrt{T}z\in D}
    \df{1}{\sqrt{2\pi}}e^{-(z-\sigma \sqrt{T})^2/2} dz
    -K e^{-rT} \int_{\sqrt{T}z\in D}
    \df{1}{\sqrt{2\pi}}e^{-z^2/2} dz \\
    &=S_0 \int_{\sqrt{T}(y+\sigma\sqrt{T})\in D}\phi(y)dy
    -K e^{-rT} \int_{\sqrt{T}z\in D}\phi(z)dz
  \end{split}
\end{align*}
ここで,
\begin{align*}
  D=\{x;S_T(x)-K>0\}=\left\{x; x>\df{1}{\sigma} \left[\ln{\df{K}{S_0}}
  -\left(r-\df{\sigma^2}{2}\right) T \right] \right\}
\end{align*}
であるから,最終行の$y,z$の積分範囲はそれぞれ,
\begin{align*}
  &y>\df{1}{\sigma\sqrt{T}} \left[\ln{\df{K}{S_0}}
  -\left(r-\df{\sigma^2}{2}\right) T \right] - \sigma\sqrt{T}
  %=\df{1}{\sigma\sqrt{T}} \left[\ln{\df{K}{S_0}}
  %-\left(r+\df{\sigma^2}{2}\right) T \right]
  =-\df{1}{\sigma\sqrt{T}} \left[\ln{\df{S_0}{K}}
  +\left(r+\df{\sigma^2}{2}\right) T \right] \\
  & z>\df{1}{\sigma\sqrt{T}} \left[\ln{\df{K}{S_0}}
  -\left(r-\df{\sigma^2}{2}\right) T \right]
  =-\df{1}{\sigma\sqrt{T}} \left[\ln{\df{S_0}{K}}
  +\left(r-\df{\sigma^2}{2}\right) T \right]
\end{align*}
であり,
\begin{align*}
  d_{\pm}(T,S_0)=\df{1}{\sigma\sqrt{T}} \left[\ln{\df{S_0}{K}}
  +\left(r\pm\df{\sigma^2}{2}\right) T \right]
\end{align*}
とおけば,
\begin{align*}
  \E[e^{-rT}(S_T-K)^+]=S_0 N(d_+(T,S_0))
  -K e^{-rT} N(d_-(T,S_0)).
\end{align*}

\subsection{} %3.6
\subsubsection{(i)}
ドリフトを持つブラウン運動$X_t=\mu t+W_t$について,
\begin{align*}
  \begin{split}
    \E[f(X_t)|\F(s)]&=\E[f(\mu t+W_t)|\F(s)]
    =\E[f(W_t-W_s+W_s+\mu t)|\F(s)] \\
    &=\intinf f(w+W_s+\mu t)\df{1}{\sqrt{2\pi(t-s)}}
    \exp\left(-\frac{w^2}{2(t-s)}\right)dw \\
    &=\intinf f(y)\df{1}{\sqrt{2\pi(t-s)}}
    \exp\left(-\frac{(y-W_s-\mu t)^2}{2(t-s)}\right)dy\\
    &=\intinf f(y)\df{1}{\sqrt{2\pi(t-s)}}
    \exp\left(-\frac{(y-W_s-\mu s-\mu(t-s))^2}{2(t-s)}\right)dy\\
    &=\intinf f(y)\df{1}{\sqrt{2\pi(t-s)}}
    \exp\left(-\frac{(y-X_s-\mu(t-s))^2}{2(t-s)}\right)dy\\
    &=\intinf f(y)p(t-s,X_s,y)dy=g(X_s).
  \end{split}
\end{align*}

\subsubsection{(ii)}
$\mu=\nu/\sigma$とすれば,
幾何ブラウン運動は,
$S_t=S_0 e^{\sigma W_t+\nu t}=S_t=S_0 e^{\sigma X_t}$とかける.
(i)の結果(5行目)を利用して,
\begin{align*}
  \begin{split}
    \E[f(S_t)|\F(s)]&=\E[f(S_0 e^{\sigma X_t})|\F(s)] \\
    &=\intinf f(S_0 e^{\sigma y})\df{1}{\sqrt{2\pi(t-s)}}
    \exp\left(-\frac{(y-X_s-\mu(t-s))^2}{2(t-s)}\right)dy
  \end{split}
\end{align*}
$z=S_0 e^{\sigma y}=S_0 e^{\sigma X_t}$という変数変換を考えれば,
積分範囲は$z:0\to\infty$で,\\
$y=\df{1}{\sigma}\ln{\df{z}{S_0}}, \quad dy=\df{1}{\sigma z}dz, \quad
y-X_s=X_t-X_s=\df{1}{\sigma}(\ln{\df{z}{S_0}}-\ln{\df{S_s}{S_0}})
=\df{1}{\sigma}\ln{\df{z}{S_s}}$であるから,
\begin{align*}
  \begin{split}
    \E[f(S_t)|\F(s)]
    &=\intzero f(z)\df{1}{\sqrt{2\pi(t-s)}}
    \exp\left\{-\frac{\left( \df{1}{\sigma}\ln\df{z}{S_s}
    -\mu(t-s) \right)^2}{2(t-s)}\right\}\df{1}{\sigma z}dz \\
    &=\intzero f(z)\df{1}{\sigma z\sqrt{2\pi(t-s)}}
    \exp\left\{-\frac{\left(\ln(z/S_s)-\nu(t-s)\right)^2}
    {2\sigma^2(t-s)}\right\}dz \\
    &=\intzero f(z) p(t-s,S_s,z)dz=g(S_s).
  \end{split}
\end{align*}

\subsection{}
\subsubsection{(i)}
$\E[Z_t|\F_s]=\E[Z_s]$を示したいが,
\begin{align*}
  \begin{split}
     \E\left[\df{Z_t}{Z_s}|\F_s\right]
     &=\E\left[\exp\left\{ \sigma(X_t-X_s)
     -(\sigma\mu+\df{1}{2}\sigma^2)(t-s) \right\}|\F_s\right]\\
     &=\E\left[\exp\left\{\sigma\mu(t-s)+\sigma(W_t-W_s)
     -(\sigma\mu+\df{1}{2}\sigma^2)(t-s)\right\}|\F_s\right]\\
     &=\E\left[\exp\left\{\sigma(W_t-W_s)\right\}\right]
     \cdot\exp\left\{-\df{1}{2}\sigma^2(t-s)\right\}=1.
   \end{split}
\end{align*}
を得る.従って,$Z_t,t\geq0$はマルチンゲール.

\subsubsection{(ii)}
$Z_t$はマルチンゲールであるから,任意抽出定理より,
\begin{align*}
  \E[Z_{t\wedge\tau_m}]=\E[Z_0]=\E[\exp(0)]=1, \quad
  t\geq0.
\end{align*}

\subsubsection{(iii)}
$\tau_m=\min\{t\geq0; \, X_t=m\}$であるから,$t\leq\tau_m$において,
以下が常に成り立つ.
\begin{align*}
  0 \leq \exp\{\sigma X_{t\wedge\tau_m}\} \leq e^{\sigma m}.
\end{align*}

まずは,
$\exp\{-(\sigma\mu+\frac{\sigma^2}{2})(t\wedge\tau_m)\}$
について考える.\\
$\{\tau_m<\infty\}$のとき,
\begin{align*}
  \lim_{t\to\infty}
  \exp\{-(\sigma\mu+\frac{\sigma^2}{2})(t\wedge\tau_m)\}
  = \exp\{-(\sigma\mu+\frac{\sigma^2}{2})\tau_m\},
\end{align*}
$\{\tau_m=\infty\}$のとき,
\begin{align*}
  \exp\{-(\sigma\mu+\frac{\sigma^2}{2})(t\wedge\tau_m)\}
  = \exp\{-(\sigma\mu+\frac{1}{2}\sigma^2)t\}
  \to0 \quad (t\to\infty),
\end{align*}
これらをまとめると,以下のように表せる.
\begin{align*}
  \lim_{t\to\infty}
  \exp\{-(\sigma\mu+\frac{\sigma^2}{2})(t\wedge\tau_m)\}
  = \1_{\{\tau_m<\infty\}}
  \exp\{-(\sigma\mu+\frac{\sigma^2}{2})\tau_m\}.
\end{align*}

次に,$\exp\{\sigma X_{t\wedge\tau_m}\}$について考える.\\
$\{\tau_m<\infty\}$のとき,
\begin{align*}
  \lim_{t\to\infty}
  \exp\{\sigma X_{t\wedge\tau_m}\}
  = e^{\sigma X_{\tau_m}}=e^{\sigma m},
\end{align*}
$\{\tau_m=\infty\}$のとき,
$t\to\infty$で$\exp\{\sigma X_{t\wedge\tau_m}\}$
が有界であることは保証できる.
したがって,$\exp\{-(\sigma\mu+\frac{\sigma^2}{2})(t\wedge\tau_m)\}$
と$\exp\{\sigma X_{t\wedge\tau_m}\}$の積の極限値($t\to\infty$)は
ゼロである.

以上の議論をまとめると,以下を得る.
\begin{align*}
  \begin{split}
    \lim_{t\to\infty} Z_{t\wedge\tau_m}
    &=\lim_{t\to\infty}
    \exp\{ \sigma X_{t\wedge\tau_m}
    -(\sigma\mu+\frac{\sigma^2}{2})(t\wedge\tau_m) \} \\
    &=\1_{\{\tau_m<\infty\}} \exp\{ \sigma m
    -(\sigma\mu+\frac{\sigma^2}{2})\tau_m \} \\
    &=\1_{\{\tau_m<\infty\}}Z_{\tau_m}
  \end{split}
\end{align*}

3.7(ii)の結果について極限をとると,優収束定理により,
\begin{align*}
  \lim_{t\to\infty}\E[Z_{t\wedge\tau_m}]
  =\E[\lim_{t\to\infty}Z_{t\wedge\tau_m}]
  =\E[\1_{\{\tau_m<\infty\}}Z_{\tau_m}]
  =1.
\end{align*}
つまり,
\begin{align*}
  \E\left[\1_{\{\tau_m<\infty\}} \exp\{ \sigma m
  -(\sigma\mu+\frac{\sigma^2}{2})\tau_m \} \right]=1
\end{align*}
を得る(前半部証明終了).これと同値な式として次式を得る.
\begin{align*}
  \E\left[\1_{\{\tau_m<\infty\}}
  \exp\{-(\sigma\mu+\frac{\sigma^2}{2})\tau_m\}\right]
  =e^{-\sigma m}.
\end{align*}
全ての正の$\sigma$でこの式は成り立つので,
両辺を$\sigma\to0$として極限をとると,単調収束定理より,
$\E[\1_{\{\tau_m<\infty\}}]=1$を得る.それと同値な式として,
$\Pr\{\tau_m<\infty\}=1$を得る.
この結果から,上で得た期待値の式の定義関数を外すことができ,
\begin{align*}
  \E\left[\exp\{-(\sigma\mu+\frac{\sigma^2}{2})\tau_m\}\right]
  =e^{-\sigma m}.
\end{align*}
を得る.最後に,
$\sigma\mu+\frac{\sigma^2}{2}=\alpha$とおけば,
$\sigma>0,\alpha>0$という条件の下で$\sigma$について解くと,
\begin{align*}
  \sigma=-\mu+\sqrt{2\alpha+\mu^2}
\end{align*}
となり,以下のラプラス変換を得る.
\begin{align*}
  \E[e^{-\alpha\tau_m}]=e^{-\sigma m}
  =e^{m\mu-m\sqrt{2\alpha+\mu^2}}, \quad \alpha>0.
\end{align*}

\subsection{} %3.8
\subsubsection{(i)}
$\frac{1}{\sqrt{n}}M_{nt,n}$のモーメント母関数$\varphi(u)$は,
リスク中立測度を用いて以下のように表せる.
\begin{align*}
  \begin{split}
    \varphi_n(u)
    &=\tilde{\E}\left[\exp(u\frac{1}{\sqrt{n}}M_{nt,n})\right]
    =\tilde{\E}\left[\exp\left\{\frac{u}{\sqrt{n}}
    (X_{i,n}+...+X_{nt,n})\right\} \right] \\
    &=\left( \tilde{\E}\left[e^{\frac{u}{\sqrt{n}}X_{1,n}}
    \right] \right)^{nt}
    =\left( e^{\frac{u}{\sqrt{n}}\cdot 1}\tilde{p}_n
    +e^{-\frac{u}{\sqrt{n}}\cdot (-1)}\tilde{q}_n \right)^{nt}
    \\
    &=\left[ e^{\frac{u}{\sqrt{n}}}
    \left(  \df{\frac{r}{n}+1-e^{-\sigma/\sqrt{n}}}
    {e^{\sigma/\sqrt{n}}-e^{-\sigma/\sqrt{n}}}  \right)
    +e^{-\frac{u}{\sqrt{n}}}
    \left(  \df{e^{\sigma/\sqrt{n}}-\frac{r}{n}-1}
    {e^{\sigma/\sqrt{n}}-e^{-\sigma/\sqrt{n}}}  \right)
    \right]^{nt} \\
    &=\left[ e^{\frac{u}{\sqrt{n}}}
    \left(  \df{\frac{r}{n}+1-e^{-\sigma/\sqrt{n}}}
    {e^{\sigma/\sqrt{n}}-e^{-\sigma/\sqrt{n}}}  \right)
    -e^{-\frac{u}{\sqrt{n}}}
    \left(  \df{\frac{r}{n}+1-e^{\sigma/\sqrt{n}}}
    {e^{\sigma/\sqrt{n}}-e^{-\sigma/\sqrt{n}}}  \right)
    \right]^{nt}.
  \end{split}
\end{align*}

\subsubsection{(ii)}
変数変換$x=\frac{1}{\sqrt{n}}$を施すと,
\begin{align*}
  \varphi_{\frac{1}{x^2}}(u)
  =\left[ e^{ux}
  \left(  \df{rx^2+1-e^{-\sigma x}}
  {e^{\sigma x}-e^{-\sigma x}}  \right)
  -e^{-ux}
  \left(  \df{rx^2+1-e^{\sigma x}}
  {e^{\sigma x}-e^{-\sigma x}}  \right)
  \right]^{\frac{t}{x^2}}
\end{align*}
を得る.
対数をとると,次式を得る.
\begin{align*}
  \begin{split}
    \ln\varphi_{\frac{1}{x^2}}(u)
    &=\df{t}{x^2}\ln\left[\df {(rx^2+1)(e^{ux}-e^{-ux})
    +e^{(\sigma-u)x}-e^{-(\sigma-u)x}}
    {e^{\sigma x}-e^{-\sigma x}} \right] \\
    &=\df{t}{x^2}\ln\left[
    \df {(rx^2+1) \sinh{ux} +\sinh{(\sigma-u)x}}
    {\sinh{\sigma x}} \right] \\
    &=\df{t}{x^2}\ln\left[
    \df {(rx^2+1) \sinh{ux}
    + \sinh{\sigma x}\cosh{ux} - \cosh{\sigma x}\sinh{ux} }
    {\sinh{\sigma x}} \right] \\
    &=\df{t}{x^2}\ln\left[ \cosh{ux}
    +\df{(rx^2+1-\cosh{\sigma x}) \sinh{ux}}
    {\sinh{\sigma x}} \right].
  \end{split}
\end{align*}

\subsubsection{(iii)}
$x$の極限をとることに注意すれば,テイラー級数展開を用いて
次式を得る.
\begin{align*}
  \begin{split}
    &\cosh{ux}+\df{(rx^2+1-\cosh{\sigma x})\sinh{ux}}
    {\sinh{\sigma x}} \\
    =&1+\df{u^2 x^2}{2}+O(x^4)+
    \df{ \left( rx^2+1 -1-\frac{\sigma^2 x^2}{2} +O(x^4)
    \right) (ux+O(x^3)) } {\sigma x+O(x^3)} \\
    =&1+\df{u^2 x^2}{2}+O(x^4)+
    \df{\left( rx^2-\frac{\sigma^2 x^2}{2}\right)ux^3+O(x^5)}
    {\sigma x+O(x^3)} \\
    =&1+\df{u^2 x^2}{2}+
    \df{\left( rx^2-\frac{\sigma^2 x^2}{2}\right)
    ux^3(1+O(x^2))}  {\sigma x(1+O(x^2))}  +O(x^4) \\
    =&1+\df{u^2 x^2}{2}+\df{rux^2}{\sigma}
    -\df{\sigma ux^2}{2}+O(x^4).
  \end{split}
\end{align*}

\subsubsection{(iv)}
3.8(iii)の結果より,$\ln\varphi_{\frac{1}{x^2}}(u)$
に対してもテイラー級数展開を用いて次式を得る.
\begin{align*}
  \ln\varphi_{\frac{1}{x^2}}(u)
  =\df{t}{x^2} \ln\left(
  1+\df{u^2 x^2}{2}+\df{rux^2}{\sigma}
  -\df{\sigma ux^2}{2}+O(x^4) \right)
  =\df{t}{x^2}\left(\df{u^2 x^2}{2}+\df{rux^2}{\sigma}
  -\df{\sigma ux^2}{2}+O(x^4) \right).
\end{align*}
$x\to0$の極限をとれば,
\begin{align*}
  \lim_{x\to0}\ln\varphi_{\frac{1}{x^2}}(u)
  =t(\df{u^2}{2}+\df{ru}{\sigma}-\df{\sigma u}{2})
  =\df{1}{2}tu^2+t(\df{r}{\sigma}-\df{\sigma}{2})u
\end{align*}
を得る.$x\to0$は$n\to\infty$の極限に対応するため,
これは,二項モデルの確率変数$\frac{1}{\sqrt{n}}M_{nt,n}$の
極限分布を考えていることにあたる.
モーメント母関数に対して確率分布は一意に定まり,
上で得たキュムラント(モーメント母関数の対数)から,
$\frac{1}{\sqrt{n}}M_{nt,n}$の極限分布は
平均$t(\frac{r}{\sigma}-\frac{\sigma}{2})$,
分散$t$の正規分布となることがわかる.
したがって,$\frac{\sigma}{\sqrt{n}}M_{nt,n}$の極限分布は,
平均$t(r-\frac{\sigma^2}{2})$,
分散$\sigma^2 t$の正規分布となる.

\subsection{} %3.9
考え中.
\end{document}
